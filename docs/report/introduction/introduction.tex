\section{Résumé}

Dans notre projet nous allons chercher à modéliser une ville en 3D grâce à l'outil Unity, le but de notre création sera de :
\begin{itemize}
	\item Créer un terrain en relief avec une surface plus plane pour placer la ville
	\item Créer une ville entourée d'une muraille pour obtenir un aspect de ville ancienne, ou de ville médiévale
	\item La ville devra être composé de bâtiments le long des routes et à l'intérieur de la muraille
	\item On doit donc aussi y retrouver des routes qui pour les principales sortiront de la ville par des portes situées sur la muraille
	\item Dans la partie extérieure de la ville on doit pouvoir voir aussi des rivières
	\item La création de la ville doit pouvoir être manipulable par une interface pour établir la taille de la ville ou le nombre de bâtiments.
\end{itemize}

\section{Introduction}

Le projet de génération de villes aléatoires consiste à créer une carte d’une ville à partir d'un plan vide, modélisant ainsi les routes, les bâtiments, et d’autres infrastructures, sur une carte
ayant des spécificités environnementales, elles-mêmes générées par le programme.
Pour mettre à bien notre projet, nous allons utiliser des ressources tel que :

\tab - L’article de CityGen et La thèse de George Kelly sur la génération de villes aléatoires

\tab - L’application Toy Town de Watabou
\subsection{Les objectifs}
\begin{itemize}
    
  \item Génération d'un terrain 3D possédant des collines, montagnes, rivières, lacs.
  \item Création de routes principales qui connectent la ville à l'extérieur des murailles, avec des petites routes annexes au sein de la ville et à l'extérieur.
  \item Les routes s'adaptent au terrain, elle doivent être le plus possible sur des surfaces planes.
  \item Une muraille entoure la ville avec des portes dans des endroits stratégiques.
  \item Des places, jardins (au milieu de la ville préférablement).
  \item Plusieurs types de bâtiments : château, ferme, hôtel de ville, mairie, école, qui sont générés grâce a l'algorithme de placement de structures clés, vers le centre de la ville.
  \item On peut avoir des cellules entre les routes où poser des bâtiments, les cellules peuvent être de plusieurs formes, d'un triangle (3 côtés) à un hexagone (6 côtés).
\end{itemize}

\subsection{Les utilisations}

L’utilisation d’un générateur de villes pourrait être utile pour les entreprises cherchant à créer de nouvelles infrastructures dans des villes déjà existantes, ou pour modéliser par exemple des routes sur une carte générée par le programme.

La génération d'une ville médiévale pourrait être utile aux chercheurs dans le domaine historique cherchant à créer une représentation d'une ville ancienne.

Créer des villes est également très utile dans le domaine des jeux vidéos pour modéliser les maps des jeux.